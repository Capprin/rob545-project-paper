\documentclass[10pt,conference]{ieeeconf}
\usepackage{amsmath}
\usepackage{amssymb}
\usepackage{cite}
\usepackage{graphicx}
\usepackage{mathnotation}
\usepackage{oubraces}

\makeatletter
\def\endthebibliography{%
  \def\@noitemerr{\@latex@warning{Empty `thebibliography' environment}}%
  \endlist
}
\makeatother

\DeclareRobustCommand{\groupderiv}[1]{\accentset{\scriptstyle\circ}{#1}}
\renewcommand{\localconn}{\mixedconn}
% taken from IEEEtran
\DeclareRobustCommand*{\IEEEauthorrefmark}[1]{\raisebox{0pt}[0pt][0pt]{\textsuperscript{\footnotesize\ensuremath{\ifcase#1\or *\or \dagger\or \ddagger\or%
    \mathsection\or \mathparagraph\or \|\or **\or \dagger\dagger%
    \or \ddagger\ddagger \else\textsuperscript{\expandafter\romannumeral#1}\fi}}}}

\title{Kinematic Path Planning Under Uncertainty}
% Optimal Choice of Robot Intrinsic Dimension
\author{
    \authorblockN{
        Capprin Bass\IEEEauthorrefmark{1},
        Neha Pusalkar\IEEEauthorrefmark{2}, and
        Brett Stoddard\IEEEauthorrefmark{3}
    }
    \authorblockA{  
        Collaborative Institute for Robotics and Intelligent Systems (CoRIS),
        Oregon State University\\Corvallis, Oregon \\
        Email: \{\IEEEauthorrefmark{1}basscap,
        \IEEEauthorrefmark{2}pusalkan,
        \IEEEauthorrefmark{3}stoddabr\}@oregonstate.edu
    }
}

\begin{document}

\maketitle

\begin{abstract}
    Recently in the robotics industry, companies construct certain research robots from cheaper components, lowering the barrier of entry to study robot systems.
	These systems pose a challenge from a planning and control standpoint, as uncertainty in actuation and sensing propagates into the trajectory of the robot.
	In this paper, we present a novel method of expressing joint space uncertainty in the task space, respecing system kinematics.
	We cast this expression for uncertainty as a pathlength metric, reflecting uncertainty accrued over a path.
	Finally, we use our covariant pathlength metric as a heuristic for path planning, producing paths for a planar manipulator that minimize uncertainty.
\end{abstract}

\section{Introduction}\label{sec:introduction}
% historic robots
	% powerful actuators
	% precise encoders
	% task space plans can be followed using simple kinematics or dynamics
Modern industrial robots are built to be controlled using kinematic and dynamic principles.
Many systems take advantage of powerful actuators, precise sensors, and fast computers to make software control tenable, using these mechanical models.
Task space path plans can be followed with PID control on top of inverse kinematics or dynamics; the result is a robust stack of methods that has seen widespread use in industry \cite{something?}
However, this approach literally comes at a cost: industrial robots range in price from tens to hundreds of thousands of dollars \cite{pricing}

% newer robots
	% goal of lower cost / barrier to entry
	% cheaper actuators + hardware
	% more sources of error, so uncertainty ought to be accounted for in planning & control
A recent trend in robotics is the construction of lower cost systems, often with the goal of a lower barrier to entry in robotics research.
Some current examples of robots built with this goal include Hello Robot's Stretch \cite{stretch}, Pollen Robotics' Reachy \cite{reachy} and others, as well as a universe of hobby and DIY robots.
These systems often may be characterized as the antithesis of current industrial robots: inexpensive actuators, sensors, and computers are chosen on purpose to keep cost low.
The relative imprecision of constitutent components makes control of these systems a challenge.
Uncertainty must be built into models for system behavior, and ought to be accounted for when designing motion plans.

% here, provide a simple approach to represent first order error in robot systems, based on system kinematics
	% take covariance matrix defined on joint space as first order representation of error
	% use first order kinematic map (jacobian) to represent variance in the task space
	% provide Jinv expression
	% use task space variance as a metric on path length
	% incorporate in path planning to find low-variance paths between points
In this paper, we address the need for uncertainty-aware path planning with a novel approach, taking advantage of first-order system kinematics to represent uncertainty at the end effector.
Unlike many previous approaches \cite{previous work?}, our method respects the mapping between joint space uncertainty and the task space, which provides an intuitive representation of uncertainty at any configuration.
We use the covariance between actuators to represent uncertainty in the joint space and map it up to the task space using an \textit{inverse pullback} operation:
\begin{equation}
	M = \left(\inv{J}\right)^T \Sigma \; \inv{J},
\end{equation}
where $M$ is the uncertainty expressed in the task space, $J$ is the jacobian of the manipulator, and $\Sigma$ is the joint-space defined covariance.
The task space uncertainty $M$ may be used as a length metric for paths defined in the task space; it reflects the uncertainty accrued by taking a given path.
We demonstrate our covariant pathlength metric as a heuristic for path planning, producing paths that respect system uncertainty.

% roadmap
The remainder of this paper is organized as follows.
In \S\ref{sec:background}, we review the relevant background in kinematics, geometry, and path planning.
In \S\ref{sec:methods}, we describe how the covariance matrix is mapped into the task space, and how uncertainty is accounted for in path planning.
In \S\ref{sec:analysis}, we demonstrate our method on a planar manipulator, generating low-uncertainty paths.
In \S\ref{sec:conclusion}, we make final remarks and comment on future work.

\section{Background}\label{sec:background}
\subsection{Kinematic Maps}
% define joint space, task space
% jacobian maps between tangent spaces
% how to do inverse pullback of a bilinear form
\subsection{Riemannian Metrics}
\subsection{Path Planning}
% brief explanation of A* search
% importance of distance measures

\section{Covariance as a Pathlength Metric}\label{sec:methods}
\subsection{Kinematic Mapping of the Covariance Matrix}
% definition of covariance matrix
% inverse pullback of the covariance matrix
% interpretation in the task space (figure)
\subsection{Planning Under Uncertainty}
% given a task space path, how do we construct the covariance matrix?
% perform newton-rhapson inverse kinematics at each point to get the Jacobian
% compute covariance matrix at each point
% use cov mat to compute length (show math)
% include algorithm environment
% use as heuristic/pathlength in planning algorithms

\section{Results and Analysis}\label{sec:analysis}

\section{Conclusion}\label{sec:conclusion}

\bibliographystyle{./IEEEtran}
\bibliography{./IEEEabrv,./gm-references/gm.bib}

\end{document}
